\documentclass[UTF8]{ctexart}
\usepackage{amsmath}   % 数学公式
\usepackage{booktabs}  % 三线表
\usepackage{geometry}  % 页面设置
\geometry{a4paper, left=2.5cm, right=2.5cm, top=2.5cm, bottom=2.5cm}

\begin{document}

\title{计算流体力学第二次作业}
\author{朱林-2200011028}
\date{\today}
\maketitle

\section{数理算法原理}

\subsection{对于一阶导数$\frac{\partial u}{\partial x}$的差分格式}

\subsubsection{采用两个网格点的一阶格式}
利用泰勒展开式,对于$u_{i+1}$,我们有:
\begin{equation}
    u_{i+1} = u_i + \Delta x \frac{\partial u}{\partial x} + O(\Delta x^2)
\end{equation}
在等式两侧同时除以$\Delta x$,我们可以得到:
\begin{equation}
    \frac{u_{i+1} - u_i}{\Delta x} = \frac{\partial u}{\partial x} + O(\Delta x)
\end{equation}
由此可知,差分格式为:
\begin{equation}
    \frac{\partial u}{\partial x} = \frac{u_{i+1} - u_i}{\Delta x} 
\end{equation}
将右侧展开后观察阶数:
\begin{equation}
    \begin{aligned}
        \frac{u_{i+1} - u_i}{\Delta x} &= \frac{u_i + \Delta x \frac{\partial u}{\partial x} +  \frac{\Delta x^2}{2} \frac{\partial^2 u}{\partial x^2} + O(\Delta x^3) - u_i}{\Delta x} \\
        &= \frac{\Delta x \frac{\partial u}{\partial x} +  \frac{\Delta x^2}{2} \frac{\partial^2 u}{\partial x^2} + O(\Delta x^3)}{\Delta x} \\
        &= \frac{\partial u}{\partial x} + \frac{\Delta x}{2} \frac{\partial^2 u}{\partial x^2} + O(\Delta x^2)
    \end{aligned}
\end{equation}
由于截断误差为一阶,所以,该格式为一阶格式。

\subsubsection{采用四个网格点的三阶格式}
采用待定系数法,我们采用四个网格点来构造差分格式,设这一格式为:
\begin{equation}
   \frac{\partial u}{\partial x} = a_1 u_{i-1} + a_2 u_i + a_3 u_{i+1} + a_4 u_{i+2}
\end{equation}
将$u_{i-1}$、、$u_{i+1}$、$u_{i+2}$分别用泰勒展开式展开,得到:
\begin{equation}
    \begin{aligned}
        u_{i-1} &= u_i - \Delta x \frac{\partial u}{\partial x} + \frac{(\Delta x)^2}{2} \frac{\partial^2 u}{\partial x^2} - \frac{(\Delta x)^3}{6} \frac{\partial^3 u}{\partial x^3} + O(\Delta x^4) \\
        u_{i+1} &= u_i + \Delta x \frac{\partial u}{\partial x} + \frac{(\Delta x)^2}{2} \frac{\partial^2 u}{\partial x^2} + \frac{(\Delta x)^3}{6} \frac{\partial^3 u}{\partial x^3} + O(\Delta x^4) \\
        u_{i+2} &= u_i + 2\Delta x \frac{\partial u}{\partial x} + 2(\Delta x)^2 \frac{\partial^2 u}{\partial x^2} + \frac{4(\Delta x)^3}{3} \frac{\partial^3 u}{\partial x^3} + O(\Delta x^4)
    \end{aligned}
\end{equation}
带入(4)式中,得到:
\begin{equation}
    \begin{aligned}
        \frac{\partial u}{\partial x} &= a_1 \left( u_i - \Delta x \frac{\partial u}{\partial x} + \frac{(\Delta x)^2}{2} \frac{\partial^2 u}{\partial x^2} - \frac{(\Delta x)^3}{6} \frac{\partial^3 u}{\partial x^3} + O(\Delta x^4) \right) + a_2 u_i \\
        &+ a_3 \left( u_i + \Delta x \frac{\partial u}{\partial x} + \frac{(\Delta x)^2}{2} \frac{\partial^2 u}{\partial x^2} + \frac{(\Delta x)^3}{6} \frac{\partial^3 u}{\partial x^3} + O(\Delta x^4) \right) \\
        &+ a_4 \left( u_i + 2\Delta x \frac{\partial u}{\partial x} + 2(\Delta x)^2 \frac{\partial^2 u}{\partial x^2} + \frac{4(\Delta x)^3}{3} \frac{\partial^3 u}{\partial x^3} + O(\Delta x^4) \right)\\
        &= (a_1 + a_2 + a_3 + a_4) u_i + \left( -a_1 + a_3 + 2a_4 \right) \Delta x \frac{\partial u}{\partial x} + \left( \frac{a_1}{2} + \frac{a_2}{2} + a_3 + 2a_4 \right) (\Delta x)^2 \frac{\partial^2 u}{\partial x^2} \\
        &+ \left( -\frac{a_1}{6} + \frac{a_2}{6} + \frac{a_3}{6} + \frac{4a_4}{3} \right) (\Delta x)^3 \frac{\partial^3 u}{\partial x^3} + O(\Delta x^3)
    \end{aligned}
\end{equation}
比较两侧系数得到一下方程组:
\begin{equation}
    \begin{cases}
        a_1 + a_2 + a_3 + a_4 &= 0 \\
        \Delta x(-a_1 + a_3 + 2a_4) &= 1 \\
        \frac{a_1}{2} + \frac{a_3}{2} + 2a_4 &= 0 \\
        -\frac{a_1}{6} + \frac{a_3}{6} + \frac{4a_4}{3} &= 0
    \end{cases}
\end{equation}
解得:
\begin{equation}
    \begin{cases}
        a_1 &=  -\frac{1}{3\Delta x}\\
        a_2 &=  -\frac{1}{2\Delta x}\\
        a_3 &=  \frac{1}{\Delta x}\\
        a_4 &=  -\frac{1}{6\Delta x}
    \end{cases}
\end{equation}
因此,最终的四网格点差分格式为:
\begin{equation}
    \frac{\partial u}{\partial x} = \frac{-2u_{i-1}-3u_i + 6u_{i+1} - u_{i+2}}{6\Delta x} 
\end{equation}
同理,将右侧展开后观察阶数可以得知该格式为三阶格式。

\subsection{对于二阶导数$\frac{\partial^2 u}{\partial x^2}$的差分格式}

\subsubsection{采用三个网格点的一阶格式}
利用泰勒展开式,对于$u_{i+1}$,我们有:
\begin{equation}
    u_{i+1} = u_i + \Delta x \frac{\partial u}{\partial x} + \frac{\Delta x^2}{2} \frac{\partial^2 u}{\partial x^2} + O(\Delta x^3)
\end{equation}
对于$u_{i+2}$,我们有:
\begin{equation}
    u_{i+2} = u_i + 2\Delta x \frac{\partial u}{\partial x} + 2(\Delta x)^2 \frac{\partial^2 u}{\partial x^2} + O(\Delta x^3)
\end{equation}
利用(10)式和(11)式,消去$\frac{\partial u}{\partial x}$,我们可以得到该差分公式:
\begin{equation}
    \frac{\partial^2 u}{\partial x^2} = \frac{u_{i+2} - 2u_{i+1} + u_i}{(\Delta x)^2}
\end{equation}
将右侧展开后观察阶数:
\begin{equation}
    \begin{aligned}
        \frac{u_{i+2} - 2u_{i+1} + u_i}{(\Delta x)^2} &= \frac{1}{(\Delta x)^2}\cdot [(u_i + 2\Delta x \frac{\partial u}{\partial x} + 2(\Delta x)^2 \frac{\partial^2 u}{\partial x^2} + \frac{4(\Delta x)^3}{3} \frac{\partial^3 u}{\partial x^3} + O(\Delta x^4)) \\
        &- 2(u_i + \Delta x \frac{\partial u}{\partial x} + \frac{(\Delta x)^2}{2} \frac{\partial^2 u}{\partial x^2} + \frac{(\Delta x)^3}{6} \frac{\partial^3 u}{\partial x^3} + O(\Delta x^4)) + u_i] \\
        &= \frac{(\Delta x)^2\frac{\partial^2 u}{\partial x^2} + (\Delta x)^3 \frac{\partial^3 u}{\partial x^3} + O(\Delta x^4)}{(\Delta x)^2} \\
        &= \frac{\partial^2 u}{\partial x^2} + (\Delta x) \frac{\partial^3 u}{\partial x^3} + O(\Delta x^2)
    \end{aligned}
\end{equation}
因此,该格式为一阶格式。

\subsubsection{采用三个网格点的二阶格式}
同理,首先设定差分格式为:
\begin{equation}
    \frac{\partial^2 u}{\partial x^2} = b_1 u_{i-1} + b_2 u_i + b_3 u_{i+1}
\end{equation}
对于$u_{i-1}$、$u_{i+1}$,我们有:
\begin{equation}
    \begin{aligned}
        u_{i-1} &= u_i - \Delta x \frac{\partial u}{\partial x} + \frac{(\Delta x)^2}{2} \frac{\partial^2 u}{\partial x^2} - \frac{(\Delta x)^3}{6} \frac{\partial^3 u}{\partial x^3} + O(\Delta x^4) \\
        u_{i+1} &= u_i + \Delta x \frac{\partial u}{\partial x} + \frac{(\Delta x)^2}{2} \frac{\partial^2 u}{\partial x^2} + \frac{(\Delta x)^3}{6} \frac{\partial^3 u}{\partial x^3} + O(\Delta x^4)
    \end{aligned}
\end{equation}
带入得到:
\begin{equation}
    \begin{aligned}
        \frac{\partial^2 u}{\partial x^2} &= b_1 \left( u_i - \Delta x \frac{\partial u}{\partial x} + \frac{(\Delta x)^2}{2} \frac{\partial^2 u}{\partial x^2} - \frac{(\Delta x)^3}{6} \frac{\partial^3 u}{\partial x^3} + O(\Delta x^4) \right) \\
        &+ b_2 u_i + b_3 \left( u_i + \Delta x \frac{\partial u}{\partial x} + \frac{(\Delta x)^2}{2} \frac{\partial^2 u}{\partial x^2} + \frac{(\Delta x)^3}{6} \frac{\partial^3 u}{\partial x^3} + O(\Delta x^4) \right)\\
        &= (b_1 + b_2 + b_3) u_i + (-b_1 + b_3) \Delta x \frac{\partial u}{\partial x} + (\frac{b_1}{2} + \frac{b_3}{2}) (\Delta x)^2 \frac{\partial^2 u}{\partial x^2} \\
        &+ (-\frac{b_1}{6} + \frac{b_3}{6}) (\Delta x)^3 \frac{\partial^3 u}{\partial x^3} + O(\Delta x^2)
    \end{aligned}
\end{equation}
比较两侧系数得到以下方程组:
\begin{equation}
    \begin{cases}
        b_1 + b_2 + b_3 &= 0 \\
        -b_1 + b_3 &= 0 \\
        (\Delta x)^2 (\frac{b_1}{2} + \frac{b_3}{2}) &= 1 \\
        -\frac{b_1}{6} + \frac{b_3}{6} &= 0
    \end{cases}
\end{equation}
解得:
\begin{equation}
    \begin{cases}
        b_1 &=  \frac{1}{\Delta x^2}\\
        b_2 &=  -2\frac{1}{\Delta x^2}\\
        b_3 &=  \frac{1}{\Delta x^2}
    \end{cases}
\end{equation}
因此,最终的三网格点差分格式为:
\begin{equation}
    \frac{\partial^2 u}{\partial x^2} = \frac{u_{i-1} - 2u_i + u_{i+1}}{\Delta x^2}
\end{equation}
同理,将右侧展开后观察阶数可以得知该格式为二阶格式。


\section{代码生成与调试}


\section{结果讨论和物理解释}

\end{document}