% This is a test file for latex
% 保存为 test-chinese.tex
% 使用 XeLaTeX 或 LuaLaTeX 编译
\documentclass[UTF8]{ctexart}
\usepackage{amsmath}   % 数学公式
\usepackage{booktabs}  % 三线表
\usepackage{geometry}  % 页面设置
\geometry{a4paper, left=2.5cm, right=2.5cm, top=2.5cm, bottom=2.5cm}

\begin{document}

\title{中文 LaTeX 编译测试文档}
\author{测试员}
\date{\today}
\maketitle

\section{公式测试}

以下是爱因斯坦质能方程:
\begin{equation}
E = mc^2
\end{equation}

分情况讨论公式示例:
$$
f(x) = 
\begin{cases}
\frac{\sin x}{x}, & x \neq 0 \\
1,              & x = 0
\end{cases}
$$

多行对齐公式:
\begin{align}
(a + b)^2 &= a^2 + 2ab + b^2 \label{eq1} \\
(a - b)^2 &= a^2 - 2ab + b^2 \label{eq2}
\end{align}

\section{表格测试}

\begin{table}[htbp]
\centering
\caption{学生成绩表}
\begin{tabular}{cccc}
\toprule
姓名   & 数学 & 物理 & 总成绩 \\
\midrule
张三 & 85  & 90  & 175   \\
李四 & 92  & 88  & 180   \\
王五 & 78  & 95  & 173   \\
\bottomrule
\end{tabular}
\end{table}

\section{中文排版测试}

\subsection{标点符号测试}
中文引号测试:「你好,世界!」、“这是双引号”。破折号测试——这是一个破折号。

\subsection{混合排版测试}
当我们在讨论 $\lim_{x \to 0} \frac{\sin x}{x} = 1$ 时,需要注意中文与公式的混合排版效果。

\end{document}